\documentclass{article}
\usepackage{boilerplate}
\def\mat#1{\mathbf{#1}}
\begin{document}
\title{Magnetometer Calibration}
\section{Magnetometer Calibration}
\boilerplate{APIBACK}
{ijm@ashimaresearch.com}
{General: Anyonec interested in ellipsoidal fitting and calibration of magnetometers.}
{Linked to the lifetime of the module it describes.}
{Open, Incomplete}
\subsection{Introduction}
The calibration described in this document is designed with the goal of using
a magnetometer as a compass. While some applications using a magnetometer
may find want to measure the absolute field strength, our priority is to get a good
heading measurement and as such we would ideally seek to remove everything 
but the Earth's field.
\\
\\
In a perfect world the magnetometer would sit by itself with no
metal around it, sampling an unchanging magnetic field from the
earth with 3 perfectly balanced and orthogonal sensors.
\\
\\
In reality the magnetometer will have to deal with field modification by materials
and currents flowing near to the magnetometer. There are two major forms of error: 
1. Hard magnetic errors owing to the presence of materials like steel and/or to 
large circulating currents, both of which cause a constant offset (unless the current
is changing) and both rotate with the magnetometer; 2. Soft magnetic errors due to
the presence of materials like iron and nickel (including nickel plated parts), which
cause a stretching of the measured field.
\\
\\
The stretching and offset can be computed and removed without
recourse to a calibration magnet as long as the magnetometer can
be rotated over a full sphere in a constant, unchanging field (such as the
terrestrial field). If
there were no distortions (no stretching or offsets), the measured field vectors would all sit on a sphere
centered at the origin, since the Earth's magnetic field doesn't change while the
orientation of the magnetometer is varied. However, hard magnetic (`hard iron') errors equate 
to a non-origin center, and soft magnetic (`soft iron') errors to an ellipsoidal shape.
\\
\\
The following sections detail how to fit an ellipsoid to a set of
data points derived from rotating the magnetometer within a constant field in 
order to retrieve the parameters for a
de-calibration function $k(\vec{x})$.\\
\subsection{Some Definitions: From a Sphere to an Off-Center Ellipsoid}
Start with a unit sphere about the origin. 
\begin{equation}
\left| \vec{x} \right |^2 = \vec{x}^T \vec{x} = 1
\end{equation}
Where $\vec{x}$ is a column 3-vector. If instead the vector defines an ellipse centered on the origin and with axes aligned with the coordinate axes, then it must be scaled back to a sphere first (and keeping the equation symmetric):
\begin{equation}
(\mat{\Sigma}\vec{x})^T (\mat{\Sigma}\vec{x}) = 1
\end{equation}
\begin{equation}
\vec{x}^T \mat{\Sigma}^2 \vec{x} = 1
\end{equation}
Where $\mat{\Sigma}^2$ and $\mat{\Sigma}$ are diagonal matricies.  If we further relax the vector to one defining an ellipse that isn't oriented with the coordinate axes, then it must be rotated before being scaled:
\begin{equation}
\vec{x}^T \mat{R}^T \mat{\Sigma}^2 \mat{R} \vec{x} = 1
\end{equation}
Where $\mat{R}$ is an orthogonal rotation matrix.  Lastly, if the ellipsoid is not centered on the origin, then it has to be translated before being rotated and scaled:
\begin{equation}
(\vec{x}-\vec{c})^T \mat{R}^T \mat{\Sigma}^2 \mat{R} (\vec{x}-\vec{c}) = 1
\end{equation}
Where $\vec{c}$ is the center of the ellipsoid.\\
\\
Ignoring the center for the moment, the combined matrix $\mat{A} = \mat{B}^T\mat{B}$ with $\mat{B} = \mat{\Sigma} \mat{R}$. This equation then has the form:
\begin{equation}
\vec{x}^T \mat{A} \vec{x} = \vec{x}^T \mat{B}^T \mat{B} \vec{x} = 1
\end{equation}
$\mat{A}$ is therefore symmetric and at least positive semi-definite. $\mat{R}$ is a rotation matrix so $\mat{R}^{-1} = \mat{R}^T$ and therefor
$\mat{A} = \mat{R}^{-1}\mat{\Sigma}^2\mat{R}$. This has a similar form
to the eigen-decomposition of a squared matrix. So $\mat{A} = \mat{Q}\mat{\Lambda}\mat{Q}^{-1}$,
where $\mat{Q} = \mat{R}^{-1}$ contain eigenvectors, and $\mat{\Lambda} = \mat{\Sigma}^2$ is a
diagonal matrix of corresponding eigenvalues.
\subsection{Non-Centering of the Ellipsoid}
I will deal with the offset of the ellipsoid center from the origin by a vector
$\vec{c}$. Several parallel and equivalent approaches are shown for clarity and
for ease of comparison with other derivations.
%
\subsubsection{Method 1 - Cartesian}
In Cartesian vector form, subtracting $\vec{c}$ and multiplying through :
\begin{equation}
(\vec{x}-\vec{c})^T \mat{A} (\vec{x}-\vec{c}) = 1
\end{equation}
\begin{equation}
(\vec{x}^T-\vec{c}^T) (\mat{A}\vec{x}- \mat{A}\vec{c}) = 1
\end{equation}
\begin{equation}
\vec{x}^T \mat{A}\vec{x} - \vec{x}^T\mat{A}\vec{c}
- \vec{c}^T\mat{A}\vec{x} + \vec{c}^T\mat{A}\vec{c} = 1
\end{equation}
\begin{equation}
\vec{x}^T \mat{A}\vec{x} - 2\vec{c}^T\mat{A}\vec{x} + \vec{c}^T\mat{A}\vec{c} = 1
\end{equation}
%
\subsubsection{Method 2 - Homogeneous}
Reformulating in homogeneous co-ordinates, which requires care because the $\left\|
\vec{x} \right\|$ operator because is no longer simple: 
\begin{equation}
\left| \vec{x'}^T\vec{x'} \right| = \vec{x}^T \vec{x} + 1
\end{equation}
Where $\vec{x}'$ is the vector $\vec{x}$ augmented with an extra dimension
containing a weight, and using the notation $\left|\vec{x}\right|$ to mean 
that the homogeneous coordinates are
first scaled so the weight component is $1$. i.e. 
\begin{equation*}
\left| \left[ x ~ y ~ z ~ w \right]^T \right| = 
\left[ \frac{x}{w} ~ \frac{y}{w} ~ \frac{z}{w}
~ {1} \right]^T
\end{equation*}
\\
So:
\begin{equation} \left| \vec{x'}^T\mat{C}^T \mathcal{A}\mat{C} \vec{x'} \right| -1 = 1
\end{equation}
where $\mat{C}=\left[\begin{smallmatrix}
1 & 0 & 0 & -r\\ 0 & 1 & 0 & -s\\ 0 & 0 & 1 & -t\\ 0 & 0 & 0 & 1
\end{smallmatrix} \right]$ is the translation operator for $\vec{c}$ and $\mathcal{A}$ is
$\mat{A}$ augmented with the origin\\
$\mathcal{A}= \left[\begin{array}{c|c}
\mat{A} & 0 \\ \hline 0 & 1 \end{array}\right] = \left[\begin{smallmatrix}
A & B & C & 0\\ B & D & E & 0\\ C & E & F & 0\\ 0 & 0 & 0 & 1
\end{smallmatrix} \right]$.
%
\begin{equation}
\begin{split}
\left| \vec{x'}^T\mat{C}^T \mathcal{A} \mat{C} \vec{x'}\right| &= 2\\
\left|\vec{x'}^T\left[\begin{array}{c|c}
\mat{A} & -\mat{A}\vec{c} \\ 
\hline
-\vec{c}^T\mat{A} & \vec{c}^T \mat{A} \vec{c}+1 
\end{array} \right]\vec{x'} \right| &= 2 \\
\end{split}
\end{equation}
Multiplying out (including the weight) gives:
\begin{equation}
\begin{split}
\vec{x}^T \mat{A}\vec{x} - 2\vec{c}^T\mat{A}\vec{x} + \vec{c}^T\mat{A}\vec{c} + 1 &= 2\\
\end{split}
\end{equation}
%
\subsubsection{Method 3 - Element-wise}
Multiplying out $x^T\mat{A}x = 1$, with 
$\mat{A} = \left[ \begin{smallmatrix}
A & B & C \\
B & D & E \\
C & E & F
\end{smallmatrix} \right]$ and $\vec{x}=\left[ x,y,z \right]$ first.
\begin{equation}
Ax^2 + Dy^2 + Fz^2 + 2Bxy + 2Cxz +2Eyz = 1
\end{equation} 
Substituting $\vec{x} = \vec{x}-\vec{c}$ with $\vec{c} = \left[r,s,t\right]$
\begin{equation}
\begin{matrix}
A(x^2 - 2xr + r^2) &+& 2B(xy -xs -yr + rs) &+&\\
D(y^2 - 2ys + s^2) &+& 2C(xz -xt -zr + rt) &+&\\
F(z^2 - 2zt + t^2) &+& 2E(yz -yt -zs + st) &=&1\\
\end{matrix}
\end{equation}
Rearranging :
\begin{equation}
\begin{matrix}
Ax^2 + 2Bxy &+& Dy^2 + 2Cxz &+& Fz^2 + 2Eyz &+& \\
-2x(Ar+Bs+Ct) &+& -2y(Br+Ds+Et) &+& -2z(Cr+Es+Ft) &+&\\
Ar^2+2Brs  &+& Ds^2 + 2Crt &+& Ft^2 + 2Est &=&1  \\
\end{matrix}
\end{equation}
Which is the same form as:
\begin{equation}
\vec{x}^T \mat{A}\vec{x} - 2\vec{c}^T\mat{A}\vec{x} + \vec{c}^T\mat{A}\vec{c} = 1
\end{equation}
\subsection{The Ellipse Equation}
Let $\mat{A} = \left[ \begin{smallmatrix}
A & B & C \\
B & D & E \\
C & E & F
\end{smallmatrix} \right]$ as above, 
$R = 1 - \vec{c}^T\mat{A}\vec{c}$, and $\left[\begin{smallmatrix}
G\\H\\I \end{smallmatrix}\right] = -\mat{A}\vec{c}$
\\
Substituting and multiplying through gives :
\\
\begin{equation}
Ax^2 + Dy^2 + Fz^2 + 2Bxy + 2Cxz +2Eyz + 2Gz + 2Hy + 2Iz = R
\end{equation}
As $R$ contains unknowns but is constant for every point, we can divide it out and rewrite the problem as:
\begin{equation}
A'x^2 + D'y^2 + F'z^2 + 2B'xy + 2C'xz +2E'yz + 2G'z + 2H'y + 2I'z = 1
\end{equation}
The factor $R$ is only needed if retrieving the ellipse radii separately
is required. Otherwise this is a constant factor applied equally
to all axes, which disappears under the final normalization step.
\\
A collection of points can be least-squares fit to this equation
by constructing the linear problem
$\mat{\Omega}\vec{u} = \vec{v}$ where the i-th row of $\mat{\Omega}$ contains the factor permutations in
the above equation, $\vec{u} =
\left[A',B',...,I'\right]^T$ are the members of matrix $\mat{A}'$ as above for $\mat{A}$, and $\vec{v}=\vec{1}$.
\begin{equation}
\left[ \begin{matrix}
& & & \vdots \\
x_i^2 & 2x_iy_i & 2x_iz_i & y_i^2 & 2y_iz_i & z_i^2 & 2x_i & 2y_i & 2z_i \\
& & & \vdots  
\end{matrix}\right]
\left[ \begin{matrix} A'\\ B'\\ \vdots\\ H'\\ I' \end{matrix}\right] = \vec{1}
\end{equation}
Once found, the parameters of the ellipse can be found by using an 
eigen-decomposition of $\mat{A}'$ to reverse the above construction.
\\
\\
Note: The matrix $\mat{A'}$ being found here is symmetric only. It does
not have to be positive semi-definate and probably won't be if the
ellipse is not very close to the origin.
%
\subsection{Usage}
The target goal here is to remove the distortions caused by hard and soft magnetic errors in order to improve the use of a 3-axis magnetometer as a compass.
\\
\\
From the previous section we now have the parameters of the ellipse on which data measurements 
will sit with minimum error. From this we can remove the center offset easily.
However we want to scale the ellipse back to a sphere without changing the measured direction.
This is done by scaling along the axes of the ellipse, given by the eigen-vectors of $\mat{A}$. The result will still be noisy and so still has to be forced onto the sphere with a normalization.
\\
\\
To do this we need to rotate a measurement ($\vec{z}$), scale along the axes, and rotate back. This rotation is the same as the rotation $\mat{R}$ above in the definition of the ellipse because it also rotates back to the axes before scaling. The difference is the scale factor (it isn't squared):
\begin{equation}
\vec{z}' = \mat{R}^T \mat{\Sigma} \mat{R} \vec{z}
\end{equation}
The decalibration function is then :
\begin{equation}
k(\vec{z}) = \mat{K}\left( \vec{z} - \vec{c} \right)~\text{ where }~\mat{K} = \mat{R}^T \mat{\Sigma} \mat{R}
\end{equation}
Figure X is histogram of the angle between the corrected and uncorrected measurements when forced onto a sphere (i.e. $\hat{\vec{z}}'$ and $\hat{\vec{z}}$ respectively) from an example calibration. It shows that that angles can be off by as much as 4 degrees.
\subsection{Procedure}
\begin{enumerate}
\item Compute the matrices $\mat{\Omega}^T \mat{\Omega}$ (covariances) and $\mat{\Omega}^T \vec{1}$ (means).
\item Solve $\mat{\Omega}^T \mat{\Omega} \vec{u} = \mat{\Omega}^T \vec{1}$ for $\vec{u}$ to reconstruct $\mat{A}'$.
\item Decompose $\mat{A'}$ into $\mat{Q}\mat{\Lambda}\mat{Q}^T$.
\item Retrieve center by solving $\left[\begin{smallmatrix} G'\\H'\\I' \end{smallmatrix}\right] = -\mat{A'}\vec{c}$ for $\vec{c}$.
\item Retrieve radii by Un-scale $\mat{A'}$ back to $\mat{A}$.
\item Compute the calibration matrix.
\end{enumerate}
\subsection{Implementation}
\subsubsection{ Compute the matrices $\mat{\Omega}^T \mat{\Omega}$ and $\mat{\Omega}^T \vec{1}$ }
$\mat{\Omega}^T \vec{1}$ simply sums the elements of columns of $\mat{\Omega}$ so:
\begin{equation}
\mat{\Omega}^T \vec{1} =  \left[ \begin{matrix}
\overline { x^2 } &
2\overline { x y } &
2\overline { x z } &
\overline { y^2 } &
2\overline { y z } &
\overline { z^2 } &
2\overline { x } &
2\overline { y } &
2\overline { z }
\end{matrix}\right]^T
\end{equation}
Where $\overline{abc} = \sum_i a_i b_i c_i$ (Note: it is NOT divided by the count to get a mean).
\\
\\
$\mat{\Omega}^T\mat{\Omega}$ is more interesting. First element-wise factor out the constants in 
each row, with
$\vec{f} = 
\left[ \begin{matrix}
1 & 2 & 2 & 1 & 2 & 1 & 2 & 2 & 2
\end{matrix}\right]$, and
\begin{equation}
\mat{\Omega} = \left[ \begin{matrix}
& & & \vdots \\
x_i^2 & x_iy_i & x_iz_i & y_i^2 & y_iz_i & z_i^2 & x_i & y_i & z_i \\
& & & \vdots  
\end{matrix}\right] \otimes \left[\begin{matrix} \vdots\\ \vec{f} \\ \vdots \end{matrix}\right]
\end{equation}
\begin{equation}
\mat{\Omega}^T \mat{\Omega} = \left[ \begin{array}{cccccccccc}
\overline{x^4} & \overline{x^3y} & \overline{x^3z} & \overline{\cellcolor{green}x^2y^2} & \overline{\cellcolor{yellow}x^2yz} & \overline{\cellcolor{green}x^2z^2} & \overline{x^3} & \overline{\cellcolor{yellow}x^2y} & \overline{\cellcolor{yellow}x^2z} & \\
\overline{x^3y} & \overline{\cellcolor{green}x^2y^2} & \overline{\cellcolor{yellow}x^2yz} & \overline{xy^3} & \overline{\cellcolor{yellow}xy^2z} & \overline{\cellcolor{yellow}xyz^2} & \overline{\cellcolor{yellow}x^2y} & \overline{\cellcolor{yellow}xy^2} & \overline{\cellcolor{red}xyz} & \\
\overline{x^3z} & \overline{\cellcolor{yellow}x^2yz} & \overline{\cellcolor{green}x^2z^2} & \overline{\cellcolor{yellow}xy^2z} & \overline{\cellcolor{yellow}xyz^2} & \overline{xz^3} & \overline{\cellcolor{yellow}x^2z} & \overline{\cellcolor{red}xyz} & \overline{\cellcolor{yellow}xz^2} & \\
\overline{\cellcolor{green}x^2y^2} & \overline{xy^3} & \overline{\cellcolor{yellow}xy^2z} & \overline{y^4} & \overline{y^3z} & \overline{\cellcolor{green}y^2z^2} & \overline{\cellcolor{yellow}xy^2} & \overline{y^3} & \overline{\cellcolor{yellow}y^2z} & \\
\overline{\cellcolor{yellow}x^2yz} & \overline{\cellcolor{yellow}xy^2z} & \overline{\cellcolor{yellow}xyz^2} & \overline{y^3z} & \overline{\cellcolor{green}y^2z^2} & \overline{yz^3} & \overline{\cellcolor{red}xyz} & \overline{\cellcolor{yellow}y^2z} & \overline{\cellcolor{yellow}yz^2} & \\
\overline{\cellcolor{green}x^2z^2} & \overline{\cellcolor{yellow}xyz^2} & \overline{xz^3} & \overline{\cellcolor{green}y^2z^2} & \overline{yz^3} & \overline{z^4} & \overline{\cellcolor{yellow}xz^2} & \overline{\cellcolor{yellow}yz^2} & \overline{z^3} & \\
\overline{x^3} & \overline{\cellcolor{yellow}x^2y} & \overline{\cellcolor{yellow}x^2z} & \overline{\cellcolor{yellow}xy^2} & \overline{\cellcolor{red}xyz} & \overline{\cellcolor{yellow}xz^2} & \overline{x^2} & \overline{xy} & \overline{xz} & \\
\overline{\cellcolor{yellow}x^2y} & \overline{\cellcolor{yellow}xy^2} & \overline{\cellcolor{red}xyz} & \overline{y^3} & \overline{\cellcolor{yellow}y^2z} & \overline{\cellcolor{yellow}yz^2} & \overline{xy} & \overline{y^2} & \overline{yz} & \\
\overline{\cellcolor{yellow}x^2z} & \overline{\cellcolor{red}xyz} & \overline{\cellcolor{yellow}xz^2} & \overline{\cellcolor{yellow}y^2z} & \overline{\cellcolor{yellow}yz^2} & \overline{z^3} & \overline{xz} & \overline{yz} & \overline{z^2} & \\
\end{array}\right]
\otimes \vec{f}^T \vec{f}
\end{equation}
\\
Where $\otimes$ denotes the element wise multiplication. The 
colours highlight that the structure is more
than simply symmetric. Green shows cells that are repeated 3 times,
yellow 4 times, and red 6 times. There are 31 unique summations
in the above matrix making 34 in total. These can be accumulated
one measurement at a time.
%
\subsubsection*{Notes on floating point precision}
The summations need to be done carefully. There are power 1 through
4 factors, floats only have 24 bits of precision, and there may be
$\approx$ 100,000 points measured (2 minutes is $\approx$ 128 seconds at 1024 Hz is
131072 samples, so the count alone requires 17 bits). For more than a
few seconds of data, a summation hierarchy is needed. Additionally, 
subtracting out a good center to start with should help greatly.
%
\subsubsection*{Notes on code generation}
The rows of $\mat{A'}$, $\mat{\Omega}$ and $\mat{\Omega}^T\mat{\Omega}$ follow a construction pattern. Consider:
\begin{equation}
\begin{split}
\vec{a}= \operatorname{lo}\left(\begin{matrix}A_i=a_i\\ i \leq 3 \end{matrix}\right)
&=\left[x~y~z~1\right]\\
%-----
\vec{b}= \operatorname{lo}\left(\begin{matrix}B_{ij}=a_ia_j\\ i \leq j \leq 3 \end{matrix}\right)
&=\operatorname{lo}\left(\begin{matrix}
x^2 &     &     &   \\
xy  & y^2 &     &   \\
xz  & yz  & z^2 &   \\
x   & y   & z   & 1
\end{matrix}\right) \\
&= \left[\begin{matrix}
x^2 &
xy & y^2  &
xz & yz & z^2 & x & y & z & 1
\end{matrix}\right] \\
%-----
\vec{c}=\operatorname{lo}\left(\begin{matrix}C_{ijk}=a_ia_ja_k\\ i \leq j \leq k \leq 3 \end{matrix}\right)
&=\operatorname{lo}\left(\begin{matrix}
x^2 &     &     &   \\
xy  & y^2 &     &   \\
xz  & yz  & z^2 &   \\
x   & y   & z   & 1
\end{matrix}\right|
\left.\begin{matrix}
x^2z &      &       & \\
xyz  & y^2z &       & \\
xz^2 & yz^2 & z^3   & \\
     &      &       & 
\end{matrix}\right|
\left.\begin{matrix}
x^2y &      &  & \\
xy^2 & y^3  &  & \\
     &      &  & \\
     &      &  & 
\end{matrix}\right|
\left.\begin{matrix}
x^3 &   &   & \\
    &   &   & \\
    &   &   & \\
    &   &   & 
\end{matrix}\right)\\
&= \left[\begin{smallmatrix}
x^3  &
x^2y & xy^2 & y^3 &
x^2z & xyz  & y^2z & xz^2 & yz^2 z^3  &
x^2  & xy  & y^2 & xz  & yz  & z^2 & x  & y   & z   & 1
\end{smallmatrix}\right] \\
%----------
%\vec{x}_3 &= \operatorname{up}(\vec{x}_2^T\vec{x}_2)=\operatorname{up}
%\left(\begin{matrix}
%x^4  & x^3 y   & x^3 z   & x^3    & x^2y^2 & \dots \\
     %& x^2 y^2 & x^2 y z & x^2 y  & z^2y   & \ddots \\
     %&         & x^2 z^2 & x^2 z  & \ddots & \ddots \\
     %&         &         & \ddots & \ddots &  \\
%\end{matrix}\right)=
%\left[\begin{matrix}
%x^4 & x^3 y & x^3 z & x^3 & x^2 y^2 & \dots
%\end{matrix}\right] 
\vec{d}=\operatorname{lo}\left(\begin{matrix}D_{ijkl}=a_ia_ja_ka_l\\ i \leq j \leq k \leq l\leq 3 \end{matrix}\right)
&= \dots
\end{split}
\end{equation}
Notice that $\vec{d}$ ends in $\vec{c}$, and $\vec{c}$ ends in 
$\vec{b}$, etc.  The lengths and indices of these sequences are intimately
connected to the simplex numbers (triangular-, tetrahedral-, etc numbers):
\begin{equation}
\begin{matrix}
\triangle_1(h) = h &
\triangle_2(h) = \frac{h(h+1)}{2} &
\triangle_3(h) = \frac{h(h+1)(h+2)}{3!} &
\triangle_d(h) = \frac{(d+h-1)!}{d!(h-1)!}
\end{matrix}
\end{equation}
Where $d$ is the number of dimensions, and $h$ is the height of the simplex. Our height here is always $4$, so:
\begin{equation}
\begin{matrix}
\triangle_1(4) = 4 &
\triangle_2(4) = 10 &
\triangle_3(4) = 20 &
\triangle_4(4) = 35
\end{matrix}
\end{equation}
The indexes in the arrays $\vec{a}$, $\vec{b}$, \dots and their locations in a corresponding tensor $A_{i}$, $B_{ij}$, ... are related by :
\begin{equation}
\begin{array}{lll}
a_n = A_i      & n=\triangle_1(i) = i &  \forall i\leq 3 \\
b_n = B_{ij}   & n=\triangle_2(j)+\triangle_1(i) & \forall i\leq j \leq 3 \\
c_n = C_{ijk}  & n=\triangle_3(k)+\triangle_2(j)+\triangle_1(i) & \forall i\leq j \leq k\leq 3 \\
d_n = D_{ijkl} & n=\triangle_4(l)+\triangle_3(k)+\triangle_2(j)+\triangle_1(i) & \forall i\leq j \leq k\leq l \leq 3 \\
\end{array}
\end{equation}
This reflects the construction of simplex numbers; that the next number in a sequence (e.g. the triangular number sequence)
is the sum of all the previous numbers. The inverse of these functions requires inverting this idea; find the highest simplex number that
accounts for as much of $n$ as possible, subtract it out, and recurse for the next index down.
\begin{equation}
\begin{split}
n &= \triangle_2(j) = \frac{j^2 + j}{2} \\
\text{solve } 0 &= j^2 + j - 2n \\
\triangle_2^{-1}(n) = j &= \left\lfloor \frac{-1 \pm \sqrt{1 + 8n}}{2} \right\rfloor \\
\end{split}
\end{equation}
So:
\begin{equation}
\begin{array}{ll}
a_n = A_i      & i=\triangle_1^{-1}(n) = n \\
b_n = B_{ij}   & j=\triangle_2^{-1}(n),~ i = \triangle_1^{-1}(n-\triangle_2(j)) = n-\triangle_2(j) \\
\end{array}
\end{equation}
These relationships can be used to automatically generate $\mat{\Omega}$ and the 
unique members of $\mat{\Omega}$ and $\mat{\Omega}^{T}\mat{\Omega}$ and to 
reconstruct $\mat{A'}$, in a way that is always self-consistent.

\subsubsection{Solve $\mat{\Omega}^T \mat{\Omega} \vec{u} = \mat{\Omega}^T \vec{1}$ for $\vec{u}$}
$\mat{\Omega}^T\mat{\Omega}$ is, by construction, a positive definite symmetric
matrix and so has a unique Cholesky decomposition. However an ordinary LDL
decomposition and solve will suffice to solve for $\vec{u}$.\\
Decompose $\mat{\Omega}^T \mat{\Omega} = \mat{L}\mat{D}\mat{L}^T $ :
\begin{equation}
\begin{split}
\mat{L}\mat{D}\mat{L}^T \vec{u} &= \mat{\Omega}^T \vec{1}\\
\vec{u} &= \mat{L}^{-T}\mat{D}^{-1}\mat{L}^{-1}\mat{\Omega}^T \vec{1}\\
\end{split}
\end{equation}
By solving by back-substitution the consecutive problems:
\begin{equation}
\begin{split}
\mat{L}\vec{w} &= \mat{\Omega}^T \vec{1}\\
\mat{L}^T\vec{u} &= \mat{D}^{-1} \vec{w}\\
\end{split}
\end{equation}
\subsubsection{Eigen decompose $\mat{A'} = \mat{Q}\mat{\Lambda}\mat{Q}^{-1}$}
$\mat{A'}$ is a 3 by 3 symmetric matrix whose eigenvalues can be
solved directly from its characteristic equation. Start with:
\begin{equation}
\begin{matrix}
\det\left( \lambda \mat{I} - \mat{A'} \right) = 0 & &
\det\left( \phi \mat{I} - \mat{B} \right) = 0\\
\end{matrix}
\end{equation}
Where $\mat{A'} = p\mat{B} + q\mat{I}$, ~ $\lambda = p \phi + q$, ~ $q = \frac{1}{3}tr(\mat{A'})$ and 
$p = \sqrt{ tr \left( \frac{1}{6} (q\mat{I}-\mat{A'})^2 \right) } $.\\
\\
Then the matrix $\mat{B}$ has a characteristic equation:
\begin{equation}
\phi^3 - 3 \phi - \det(\mat{B}) = 0
\end{equation}
Which has a trigonometric solution for symmetric real matrices, via the 
substitution $\phi = 2 \cos{ \theta }$, giving the eigenvalues for $\mat{A'}$ as
\begin{equation}
\lambda_{i} = q + 2 p \cos{(\frac{1}{3}\arccos{ \left( \frac{ \det(\mat{B})}{2} \right) + \frac{2\pi}{3}i })}
\end{equation}
for $i=1,2,3$. (Note that the cyclic nature of the cosine means that $i=0$ and $i=3$ are the same, etc).
\\
\\
To retrieve the eigen vectors note that: 
\begin{equation}
(\mat{A'}-\lambda_i\mat{I})\vec{v}_i = 0
\end{equation}
implies that $\vec{v_i}$ sits in the null space of
$\mat{A'}-\lambda_i\mat{I}$, and (assuming that the eigenvalues are
distinct) that the column and rows spaces of the matrix sit in a
plane orthogonal to this eigenvector. So the cross product of any two
columns will give a vector in the null space:
\begin{equation}
[\mat{Q}]_i = \widehat{ [\mat{A'}_i]_1 \times  [\mat{A'}_i]_2 }\text{, for } \mat{A'}_i = \mat{A'} - \lambda_i\mat{I}
\end{equation}
Where $[\mat{X}]_i$ means the i-th column of $\mat{X}$.\\
\\
Lastly, note again that in this case the $\mat{Q}$s are rotation matrices and so $\mat{Q}^T = \mat{Q}^{-1}$.
\subsubsection{Retrieve center by solving for $\vec{c}$}
Defined in the construction:
\begin{equation}
\left[\begin{smallmatrix} G'\\H'\\I' \end{smallmatrix}\right] = -\mat{A'}\vec{c}
\end{equation}
Note that $G'$,$H'$,$I'$, and the elements of $\mat{A'}$ are all scaled by the constant $R$, which therefore cancels out and using the eigen-decomposition of $\mat{A'}$ :
\begin{equation}
-\left[\begin{smallmatrix} G'\\H'\\I' \end{smallmatrix}\right] = \mat{Q}\mat{\Lambda}\mat{Q}^{T}\vec{c}
\end{equation}
\begin{equation}
\vec{c} = -\mat{Q}\mat{\Lambda}^{-1}\mat{Q}^{T}\left[\begin{smallmatrix} G'\\H'\\I' \end{smallmatrix}\right] 
\end{equation}
\subsubsection{Retrieve radii by unscaling $\mat{A'}$ back to $\mat{A}$}
If the radii are needed, then $R$ has to be found so that the
eigenvalues can be rescaled. Here it is important to distinguish
between the $\mat{A}$ matrix used in the ellipse construction of
sections 2.1 and 2.2, and the $\mat{A'}=\frac{1}{R}\mat{A}$ that is retrieved in 2.3.
\begin{equation}
\begin{split}
\vec{c}^T \mat{A'} \vec{c} &= \vec{c}^T \frac{\mat{A}}{R} \vec{c} ~=~\frac{\vec{c}^T\mat{A}\vec{c}}{1 - \vec{c}^T\mat{A}\vec{c} } \\
&= \frac{1}{1 - \vec{c}^T\mat{A}\vec{c}} + \frac{-1 + \vec{c}^T\mat{A}\vec{c}}{1-\vec{c}^T\mat{A}\vec{c}}\\
&= \frac{1}{1 - \vec{c}^T\mat{A}\vec{c}} -1 = \frac{1}{R} -1\\
\end{split}
\end{equation}
So:
\begin{equation}
R = \frac{1}{\vec{c}^T\mat{A'}\vec{c} +1}\\
\end{equation}
The radii in the direction of the eigen-vectors are reciprocal of $\mat{\Sigma}$ :
\begin{equation}
\begin{split}
\mat{\Sigma} &= \sqrt{R \mat{\Lambda}}\\
&= \sqrt{ (\vec{c}^T\mat{A'}\vec{c} +1)^{-1} \mat{\Lambda}}
\end{split}
\end{equation}
\subsubsection{Usage}
The decalibration function is:
\begin{equation}
k(\vec{z}) = \mat{K}\left( \vec{z} - \vec{c} \right)
\end{equation}
Where
\begin{equation}
\mat{K} 
= \mat{R}^T \mat{\Sigma} \mat{R}
= \left\{\begin{matrix}
\mat{Q} \mat{\Sigma} \mat{Q}^T &~\text{if}~&\text{Radii are needed} \\
\mat{Q} \sqrt{\Lambda} \mat{Q}^T &~\text{if}~&\vec{z}\text{ is to be normalized anyway}\\
\end{matrix}
\right.
\end{equation}
\subsection{Matlab/Octave Proof Code}
This proof-of-concept code has been written in Matlab so that various
expressions could be checked against their Matlab equivalents. The
code has a form as close as possible to the final C/C++ (or Python)
code that cannot make use of lots of memory or complex math
libraries.
%Should be done with an include ?
\begin{verbatim}
function [radii, c, Cal] = deEllipsoid(d)
  ACC = zeros(1,48);   DD = zeros(9);  f = [ 1,2,2,1,2,1,2,2,2 ];

  for i=1:(size(d)(1)), ACC += mults( d(i,:) ); end

  DD(tril(ones(9),0)==1) = ACC(1:45);
  DD = (DD + triu(DD',1)) .* (f'*f);
  D1 = ACC(40:48).*f;

  v = DD \ D1';

  A = [ v(1) v(2) v(3) ; v(2) v(4) v(5) ; v(3) v(5) v(6) ];
  [Q, L] = eigen3x3(A);

  c = - Q * inv(L) * Q' * [ v(7) v(8) v(9) ]' ;
  r = 1 / (c' * A * c + 1) ;
  D = sqrt( L * r );
  radii = 1 ./ diag(D) ;
  Cal = Q * D * Q';
end

function OmegaOmega = mults( x )
  O          = [ ((x' * x)(tril(ones(3),0)==1))', x ];
  OmegaOmega = [ ((O' * O)(tril(ones(9),0)==1))', x ];
end

function [Q, L] = eigen3x3(A)
  q = trace(A)/3
  p = sqrt( trace( (A - q*eye(3))**2 )/6)
  B = (A - q*eye(3))/p

  theta = acos(0.5*det(B))/3
  pi23 = 2*pi/3

  l0 = q + p * 2 * cos(theta + pi23 * 3 );
  l1 = q + p * 2 * cos(theta + pi23 * 1 );
  l2 = q + p * 2 * cos(theta + pi23 * 2 );
  L = zeros(3,3)
  L([1,5,9]) = [l0,l1,l2]

  A0 = A - l0*eye(3);
  A1 = A - l1*eye(3);
  A2 = A - l2*eye(3);
  v0 = cross( A0(1,:), A0(2,:) );
  v1 = cross( A1(1,:), A1(2,:) );
  v2 = cross( A2(1,:), A2(2,:) );
  Q = [ v0'/norm(v0) , v1'/norm(v1), v2'/norm(v2) ]
end 
\end{verbatim}
\end{document}
