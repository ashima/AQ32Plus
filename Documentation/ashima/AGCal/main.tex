\documentclass{article}
\usepackage{boilerplate}
\def\mat#1{\mathbf{#1}}
\begin{document}
\title{Accelerometer and Gyro Calibration}
\section{Accelerometer and Gyro Calibration}
\boilerplate{APIBACK}
{ijm@ashimaresearch.com}
{Public interested spherical fitting and calibrating Accelerometers.}
{Linked to the lifetime of the module it describes.}
{Open, Incomplete}
\subsection{Overview}
The MPU6050 and MPU9150, accelerometer and gyro have temperature dependent biases that drift very slowly over time.
\\
The problem is how to separate the biases from without knowing the
precise orientation of the vehicle. We assume :\\
\begin{itemize}
\item the biases are linear in temperature only
\item measurement noise is Gaussian with zero mean
\item that the temperature can be varied in some way for each observation
\item the vehicle is stationary while taking a calibration observation
\item the earth gravity doesn't change between observation
\item that multiple observation can be done, each in a different orientation.
\end{itemize}
From this we can generate more relations than unknown coefficients
and so solve or fit for the coefficients.  Specifically :
The linear dependence on temperature means it must be constant
across all observations and so can be computed independently of
offsets and each observation produces a mean vector that is the sum
of the bias offset and earth gravity. Thus each observation 
generates 3 new unknowns, but 4 new relations to
add to the 3 unknown biases. So for $N$ measurements we have $3N+6$
unknowns versus $4N$ relations, and the problem becomes over
constrained for $N>6$.  (The eight mid-octant vectors would be a good choice).
To make this a linear problem though we have to do the same trick as is layed out
in detail in the Magnotometer section, accept we can assume the ellipsoid is axis
oriented, which makes this a 6 unknown problem instead of the full 9 unknowns.
\\
\subsection{Details}
For each observation the only thing changing is the temperature of
the chip (by say, using a cold spray right before the observation
starts). Data is taken continuously and accumulated to give a least
squares fit for the observation.
\\
The slope of a straight line that minimizes the square deviation
in the dependant parameter $z$ with respect to independent parameter, in this case the temperature $T$, is given by
\begin{equation}
T_z = \frac{ cov(T,z) } { cov(T,T) } = 
\frac{ (\sum_i T_i z_i) - n \bar{T}\bar{z} } { (\sum_i T_i^2) - n \bar{T}^2 } = 
\frac{ \sum_i (T_i - \bar{T})(z - \bar{z}) } { \sum_i ( T_i - \bar{T} )^2 }
\end{equation}
And the Offset can be recovered by using the slope and the means :
\begin{equation}
A_z = \bar{z} - T_z \bar{T}
\end{equation}
... tbf ...\\
\end{document}
